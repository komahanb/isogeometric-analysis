@Article{Agrawal2018,
author="Agrawal, Vishal
and Gautam, Sachin S.",
title="IGA: A Simplified Introduction and Implementation Details for Finite Element Users",
journal="Journal of The Institution of Engineers (India): Series C",
year="2018",
month="May",
day="14",
abstract="Isogeometric analysis (IGA) is a recently introduced technique that employs the Computer Aided Design (CAD) concept of Non-uniform Rational B-splines (NURBS) tool to bridge the substantial bottleneck between the CAD and finite element analysis (FEA) fields. The simplified transition of exact CAD models into the analysis alleviates the issues originating from geometrical discontinuities and thus, significantly reduces the design-to-analysis time in comparison to traditional FEA technique. Since its origination, the research in the field of IGA is accelerating and has been applied to various problems. However, the employment of CAD tools in the area of FEA invokes the need of adapting the existing implementation procedure for the framework of IGA. Also, the usage of IGA requires the in-depth knowledge of both the CAD and FEA fields. This can be overwhelming for a beginner in IGA. Hence, in this paper, a simplified introduction and implementation details for the incorporation of NURBS based IGA technique within the existing FEA code is presented. It is shown that with little modifications, the available standard code structure of FEA can be adapted for IGA. For the clear and concise explanation of these modifications, step-by-step implementation of a benchmark plate with a circular hole under the action of in-plane tension is included.",
issn="2250-0553",
doi="10.1007/s40032-018-0462-6",
url="https://doi.org/10.1007/s40032-018-0462-6"
}

